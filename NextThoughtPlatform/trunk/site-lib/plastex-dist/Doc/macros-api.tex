
\section{\module{plasTeX} --- The Python Macro and Document Interfaces\label{sec:macros-api}}

\declaremodule{standard}{plasTeX}
\modulesynopsis{The classes that make up a majority of the \plasTeX\ framework.}

While \plasTeX\ does a respectable job expanding \LaTeX\ macros, some macros
may be too complicated for it to handle.  These macros may have to be re-coded
as Python objects.  Another reason you may want to use Python-based macros
is for performance reasons.  In most cases, macros coded using Python will
be faster than those expanded as true \LaTeX\ macros.

The API for Python macros is much higher-level than that of \LaTeX\ macros.
This has good and bad ramifications.  The good is that most common forms
of \LaTeX\ macros can be parsed and processed very easily using Python code
which is easier to read than \LaTeX\ code.  The bad news is that if you 
are doing something that isn't common, you will have more work to do.
Below is a basic example.

\begin{verbatim}
from plasTeX import Command

class mycommand(Command):
    """ \mycommand[name]{title} """
    args = '[ name ] title'
\end{verbatim}

The code above demonstrates how to create a Python-based macro corresponding
to \LaTeX\ macro with the form \macro{mycommand[name]\{title\}} where `name'
is an optional argument and `title' is a mandatory argument.  In the Python
version of the macro, you simply declare the arguments in the \member{args}
attribute as they would be used in the \LaTeX\ macro, while leaving the braces
off of the mandatory arguments.  When parsed in a \LaTeX\ document, an instance
of the class \class{mycommand} in created and the arguments corresponding to
`name' and `title' are set in the \member{attributes} dictionary for that 
instance.  This is very similar to the way an XML DOM works, and there are
more DOM similarities yet to come.  In addition, there are ways to handle
casting of the arguments to various data types in Python.  The API documentation
below goes into more detail on these and many more aspects of the Python
macro API.

\subsection{Macro Objects}

\begin{classdesc}{Macro}{}
The \class{Macro} class is the base class for all Python based macros
although you will generally want to subclass from \class{Command} or 
\class{Environment} in real-world use.  There are various attributes and
methods that affect how Python macros are parsed, constructed and 
inserted into the resulting DOM.  These are described below.
\end{classdesc}

\begin{memberdesc}[Macro]{args}
specifies the arguments to the \LaTeX\ macro and their data types.  The
\member{args} attribute gives you a very simple, yet extremely powerful way
of parsing \LaTeX\ macro arguments and converting them into Python objects.
Once parsed, each \LaTeX\ macro argument is set in the \member{attributes}
dictionary of the Python instance using the name given in the \member{args}
string.  For example, the following \member{args} string will direct 
\plasTeX\ to parse two mandatory arguments, `id' and `title', and put them 
into the \member{attributes} dictonary.
\begin{verbatim}
args = 'id title'
\end{verbatim}

You can also parse optional arguments, usually surrounded by square brackets
(\lbrack~\rbrack).  However, in \plasTeX, any arguments specified in the 
\member{args}
string that aren't mandatory (i.e. no braces surrounding it) are automatically
considered optional.  This may not truly be the case, but it doesn't make
much difference.  If they truly are mandatory, then your \LaTeX\ source file will
always have them and \plasTeX\ will simply always find them even though it
considers them to be optional.  

Optional arguments in the \member{args} string are surround by matching
square brackets (\lbrack~\rbrack), angle brackets (<~>), or parentheses ((~)).
The name for the attribute is placed between the matching symbols as follows:
\begin{verbatim}
args = '[ toc ] title'
args = '( position ) object'
args = '< markup > ref'
\end{verbatim}
You can have as many optional arguments as you wish.  It is also possible to 
have optional arguments using braces (\{~\}), but this requires you
to change \TeX's category codes and is not common.

Modifiers such as asterisks (*) are also allowed in the \member{args} string.
You can also use the plus (+) and minus (-) signs as modifiers although these
are not common.  Using modifiers can affect the incrementing of counters (see
the \method{parse()} method for more information).

In addition to specifying which arguments to parse, you can also specify 
what the data type should be. 
By default, all arguments are processed and stored as document fragments.
However, some arguments may be simpler than that.  They may contain an integer,
a string, an ID, etc.  Others may be collections like a list or dictionary.
There are even more esoteric types for mostly internal use that allow you to
get unexpanded tokens, \TeX\ dimensions, and the like.  Regardless, all of 
these directives are specified in the same way, using the typecast operator:
`:'.  To cast an argument, simply place a colon (:) and the name of the 
argument type immediately after the name of the argument.  The following example
casts the `filename' argument to a string.
\begin{verbatim}
args = 'filename:str'
\end{verbatim}

Parsing compound arguments such as lists and dictionaries is very similar.
\begin{verbatim}
args = 'filenames:list'
\end{verbatim}
By default, compound arguments are assumed to be comma separated.  If you
are using a different separator, it is specified in parentheses after the type.
\begin{verbatim}
args = 'filenames:list(;)'
\end{verbatim}
Again, each element element in the list, by default, is a document fragment.
However, you can also give the data type of the elements with another typecast.
\begin{verbatim}
args = 'filenames:list(;):str'
\end{verbatim}

Parsing dictionaries is a bit more restrictive.  \plasTeX\ assumes that 
dictionary arguments are always key-value pairs, that the key is always
a string and the separator between the key and value is an equals sign (=).  
Other than that, they operate in the same manner.

A full list of the supported data types as well as more examples are
discussed in section \ref{sec:macros}.
\end{memberdesc}

\begin{memberdesc}[Macro]{argSource}
the source for the \LaTeX\ arguments to this macro.  This is a read-only 
attribute.
\end{memberdesc}

\begin{memberdesc}[Macro]{arguments}
gives the arguments in the \member{args} attribute in object form 
(i.e. \class{Argument} objects).
\note{This is a read-only attribute.}
\note{This is generally an internal-use-only attribute.}
\end{memberdesc}

\begin{memberdesc}[Macro]{blockType}
indicates whether the macro node should be considered a block-level
element.  If true, this node will be put into its own paragraph node
(which also has the \member{blockType} set to \var{True}) to make it easier 
to generate output that requires block-level to exist outside of paragraphs.
\end{memberdesc}

\begin{memberdesc}[Macro]{counter}
specifies the name of the counter to associate with this macro.  Each time
an instance of this macro is created, this counter is incremented.  
The incrementing of this counter, of course, resets any ``child'' counters
just like in \LaTeX.  By default and \LaTeX\ convention, if the macro's first 
argument is an asterisk (i.e. *), the counter is not incremented.
\end{memberdesc}

\begin{memberdesc}[Macro]{id}
specifies a unique ID for the object.  If the object has an associated 
label (i.e. \macro{label}), that is its ID.  You can also set the ID 
manually.  Otherwise, an ID will be generated based on the result of Python's
\function{id()} function.
\end{memberdesc}

\begin{memberdesc}[Macro]{idref}
a dictionary containing all of the objects referenced by ``idref'' type
arguments.  Each idref attribute is stored under the name of the 
argument in the \member{idref} dictionary.
\end{memberdesc}

\begin{memberdesc}[Macro]{level}
specifies the hierarchical level of the node in the DOM.  For most macros,
this will be set to \member{Node.COMMAND_LEVEL} or 
\member{Node.ENVIRONMENT_LEVEL} by the \class{Command} and \class{Environment}
macros, respectively.  However, there are other levels that invoke special
processing.  In particular, sectioning commands such as \macro{section} and
\macro{subsection} have levels set to \member{Node.SECTION_LEVEL} and 
\member{Node.SUBSECTION_LEVEL}.  These levels assist in the building of 
an appropriate DOM.  Unless you are creating a sectioning command or a command
that should act like a paragraph, you should leave the value of this attribute
alone.  See section \ref{sec:dom-api} for more information.
\end{memberdesc}

\begin{memberdesc}[Macro]{macroName}
specifies the name of the \LaTeX\ macro that this class corresponds to.  
By default, the Python class name is the name that is used, but there are
some legal \LaTeX\ macro names that are not legal Python class names.
In those cases, you would use \member{macroName} to specify the correct
name.  Below is an example.
\begin{verbatim}
class _illegalname(Command):
    macroName = '@illegalname'
\end{verbatim}
\note{This is a class attribute, not an instance attribute.}
\end{memberdesc}

\begin{memberdesc}[Macro]{macroMode}
specifies what the current parsing mode is for this macro.  Macro classes
are instantiated for every invocation including each \macro{begin} and 
\macro{end}.  This attribute is set to \member{Macro.MODE_NONE} for normal
commands, \member{Macro.MODE_BEGIN} for the beginning of an environment,
and \member{Macro.MODE_END} for the end of an environment.  

These attributes are used in the \method{invoke()} method to determine the
scope of macros used within the environment.  They are also used in printing
the source of the macro in the \member{source} attribute.  Unless you 
really know what you are doing, this should be treated as a read-only attribute.
\end{memberdesc}

\begin{memberdesc}[Macro]{mathMode}
boolean that indicates that the macro is in \TeX's ``math mode.''  This
is a read-only attribute.
\end{memberdesc}

\begin{memberdesc}[Macro]{nodeName}
the name of the node in the DOM.  This will either be the name given in
\member{macroName}, if defined, or the name of the class itself.
\note{This is a read-only attribute.}
\end{memberdesc}

\begin{memberdesc}[Macro]{ref}
specifies the value to return when this macro is referenced (i.e. \macro{ref}).
This is set automatically when the counter associated with the macro is
incremented.
\end{memberdesc}

\begin{memberdesc}[Macro]{source}
specifies the \LaTeX\ source that was parsed to create the object.  This
is most useful in the renderer if you need to generate an image of a
document node.  You can simply retrieve the \LaTeX\ source from this 
attribute, create a \LaTeX\ document including the source, then convert
the DVI file to the appropriate image type.
\end{memberdesc}

\begin{memberdesc}[Macro]{style}
specifies style overrides, in CSS format, that should be applied to the
output.  This object is a dictionary, so style property names are given
as the key and property values are given as the values.
\begin{verbatim}
inst.style['color'] = 'red'
inst.style['background-color'] = 'blue'
\end{verbatim}
\note{Not all renderers are going to support CSS styles.}
\end{memberdesc}

\begin{memberdesc}[Macro]{tagName}
same as \member{nodeName}
\end{memberdesc}

\begin{memberdesc}[Macro]{title}
specifies the title of the current object.  If the attributes dictionary
contains a title, that object is returned.  An \exception{AttributeError}
is thrown if there is no `title' key in that dictionary.  A title can also be 
set manually by setting this attribute.
\end{memberdesc}



\begin{methoddesc}[Macro]{digest}{tokens}
absorb the tokens from the given output stream that belong to the current
object.  In most commands, this does nothing.  However, \LaTeX\ environments
have a \macro{begin} and an \macro{end} that surround content that belong 
to them.  In this case, these environments need to absorb those tokens 
and construct them into the appropriate document object model (see the
\class{Environment} class for more information).
\end{methoddesc}

\begin{methoddesc}[Macro]{digestUntil}{tokens, endclass}
utility method to help macros like lists and tables digest their contents.
In lists and tables, the items, rows, and cells are delimited by 
\macro{begin} and \macro{end} tokens.  They are simply delimited by the
occurrence of another item, row, or cell.  This method allows you to
absorb tokens until a particular class is reached.
\end{methoddesc}

\begin{methoddesc}[Macro]{expand}{}
the \method{expand} method is a thin wrapper around the \method{invoke}
method.  The \method{expand} method makes sure that all tokens are
expanded and will not return a \var{None} value like \method{invoke}.
\end{methoddesc}

\begin{methoddesc}[Macro]{invoke}{}
invakes the macro.  Invoking the macro, in the general case, includes 
creating a new context, parsing the options of the macro, and removing 
the context.  \LaTeX\ environments are slightly different.  If 
\member{macroMode} is set to \member{Macro.MODE\_BEGIN}, the new context
is kept on the stack.  If \member{macroMode} is set to \member{Macro.MODE\_END},
no arguments are parsed, the context is simply popped.  For most macros, the
default implementation will work fine.

The return value for this method is generally \var{None} (an empty return
statement or simply no return statement).  In this case, the current object
is simply put into the resultant output stream.  However, you can also
return a list of tokens.  In this case, the returned tokens will be put 
into the output stream in place of the current object.  You can even 
return an empty list to indicate that you don't want anything to be 
inserted into the output stream.
\end{methoddesc}

\begin{methoddesc}[Macro]{locals}{}
retrieves all of the \LaTeX\ macros that belong to the scope of the
current Python based macro.
\end{methoddesc}

\begin{methoddesc}[Macro]{paragraphs}{force=True}
group content into paragraphs.  Paragraphs are grouped once all other
content has been \method{digest}ed.  The paragraph grouping routine works
like \TeX's, in that environments are included inside paragraphs.  This 
is unlike HTML's model, where lists and tables are not included inside 
paragraphs.  The \var{force} argument allows you to decide whether or not
paragraphs should be forced.  By default, all content of the node is 
grouped into paragraphs whether or not the content originally contained
a paragraph node.  However, with \var{force} set to \var{False}, a node
will only be grouped into paragraphs if the original content contained
at least one paragraph node.

Even though the paragraph method follow's \TeX's model, it is still possible
to generate valid HTML content.  Any node with the \var{blockType} attribute
set to \var{True} is considered to be a block-level node.  This means that 
it will be contained in its own paragraph node.  This paragraph node
will also have the \var{blockType} attribute set to \var{True} so that
in the renderer the paragraph can be inserted or ignored based on this
attribute.
\end{methoddesc}

\begin{methoddesc}[Macro]{parse}{tex}
parses the arguments defined in the \member{args} attribute from the given
token stream.  This method also calls several hooks as described in the table
below.

\begin{tableii}{l|l}{method}{Method Name}{Description}
\lineii{preParse()}{called at the beginning of the argument parsing process}
\lineii{preArgument()}{called before parsing each argument}
\lineii{postArgument()}{called after parsing each argument}
\lineii{postParse()}{called at the end of the argument parsing process}
\end{tableii}

The methods are called to assist in labeling and counting.  For example,
by default, the counter associated with a macro is automatically incremented
when the macro is parsed.  However, if the first argument is a modifier 
(i.e. *, +, -), the counter will not be incremented.  This is handled 
in the \method{preArgument()} and \method{postArgument()} methods.

Each time an argument is parsed, the result is put into the \member{attributes}
dictionary.  The key in the dictionary is, of course, the name given to that
argument in the \member{args} string.  Modifiers such as *, +, and - are
stored under the special key `*modifier*'.

The return value for this method is simply a reference to the 
\member{attributes} dictionary.

\note{If \method{parse()} is called on an instance with \member{macroMode}
set to \member{Macro.MODE\_END}, no parsing takes place.}
\end{methoddesc}

\begin{methoddesc}[Macro]{postArgument}{arg, tex}
called after parsing each argument. This is generally where label and
counter mechanisms are handled.

\var{arg} is the Argument instance that holds all argument meta-data
    including the argument's name, source, and options.

\var{tex} is the TeX instance containing the current context 
\end{methoddesc}

\begin{methoddesc}[Macro]{postParse}{tex}
do any operations required immediately after parsing the arguments.  This
generally includes setting up the value that will be returned when 
referencing the object.
\end{methoddesc}

\begin{methoddesc}[Macro]{preArgument}{arg, tex}
called before parsing each argument.  This is generally where label and
counter mechanisms are handled.

\var{arg} is the Argument instance that holds all argument meta-data
    including the argument's name, source, and options.

\var{tex} is the TeX instance containing the current context 
\end{methoddesc}

\begin{methoddesc}[Macro]{preParse}{tex}
do any operations required immediately before parsing the arguments.
\end{methoddesc}

\begin{methoddesc}[Macro]{refstepcounter}{tex}
set the object as the current labellable object and increment its counter.
When an object is set as the current labellable object, the next 
\macro{label} command will point to that object.
\end{methoddesc}

\begin{methoddesc}[Macro]{stepcounter}{tex}
step the counter associated with the macro
\end{methoddesc}
