
\section{\module{plasTeX.DOM} --- The \plasTeX\ Document Object Model (DOM)\label{sec:dom-api}}

\declaremodule{standard}{plasTeX.DOM}
\modulesynopsis{The Document Object Model (DOM) used by \plasTeX.}

While most \LaTeX\ processors use a stream model where the input is 
directly connected to the output, \plasTeX\ actually works in two phases.
The first phase reads in the \LaTeX\ document, expands macros, and
constructs an object similar to an XML DOM.  This object is then passed
to the renderer which translates it into the appropriate output format.
The benefit to doing it this way is that you are not limited to a single
output format.  In addition, you can actually apply multiple renderers
with only one parse step.  This section describes the DOM used by 
\plasTeX, its API, and the similarities and differences between the 
\plasTeX\ DOM and the XML DOM.

\subsection{\plasTeX\ vs. XML}

The \plasTeX\ DOM and XML DOM have more similarities than differences.
This similarity is purely intentional to reduce the learning curve and
to prevent reinventing the wheel.  However, the XML DOM can be a bit 
cumbersome especially when you're used to much simpler and more elegant
Python code.  Because of this, some Python behaviors were adopted into
the \plasTeX\ DOM.  The good news is that these extensions do not break
compatibility with the XML DOM.  There are, however, some differences 
due to conventions used \LaTeX.  

The only significant difference between the \plasTeX\ DOM and the XML DOM is 
that \plasTeX\ nodes do not have true attributes like in XML.  Attributes in XML
are more like arguments in \LaTeX, because they are similar the \plasTeX\
DOM actually puts the \LaTeX\ macro arguments into the \member{attributes}
dictionary.  This does create an incompatibility though since XML DOM
attributes can only be strings whereas \LaTeX\ arguments can contain 
lots of markup.  In addition, \plasTeX\ allows you to convert these 
arguments into Python strings, lists, dictionaries, etc., so essentially
any type of object can occur in the \member{attributes} dictionary.

Other than paying attention to the the attributes dictionary difference,
you can use most other XML DOM methods on \plasTeX\ document objects to
create nodes, delete nodes, etc.  The full API is described below.

In most cases, you will not need to be concerned with instantiating nodes.
The \plasTeX\ framework does this.  However, the API can be helpful if you
want to modify the document object that \plasTeX\ creates.


\subsection{Node Objects}

\begin{classdesc}{Node}{}
The \class{Node} class is the base class for all nodes in the
\plasTeX\ DOM inluding elements, text, etc.
\end{classdesc}

\begin{memberdesc}[Node]{attributes}
a dictionary containing the attributes, in the case of \plasTeX\, the 
\LaTeX\ macro arguments
\end{memberdesc}

\begin{memberdesc}[Node]{childNodes}
a list of the nodes that are contained by this one.  In \plasTeX, this 
generally contains the contents of a \LaTeX\ environment.
\end{memberdesc}

\begin{memberdesc}[Node]{isElementContentWhitespace}
boolean indicating whether or not the node only contains whitespace.
\end{memberdesc}

\begin{memberdesc}[Node]{lastChild}
the last node in the \member{childNodes} list.  If there are no child nodes,
the value is \var{None}.
\end{memberdesc}

\begin{memberdesc}[Node]{nodeName}
the name of the node.  This is either the special node name as specified
in the XML DOM (e.g. \#document-fragment, \#text, etc.), or, if the
node corresponds to an element, it is the name of the element.
\end{memberdesc}

\begin{memberdesc}[Node]{nodeType}
integer indicating the type of the node.  The node types are defined as:
\begin{itemize}
\item \member{Node.ELEMENT\_NODE}
\item \member{Node.ATTRIBUTE\_NODE}
\item \member{Node.TEXT\_NODE}
\item \member{Node.CDATA\_SECTION\_NODE}
\item \member{Node.ENTITY\_REFERENCE\_NODE}
\item \member{Node.ENTITY\_NODE}
\item \member{Node.PROCESSING\_INSTRUCTION\_NODE}
\item \member{Node.COMMENT\_NODE}
\item \member{Node.DOCUMENT\_NODE}
\item \member{Node.DOCUMENT\_TYPE\_NODE}
\item \member{Node.DOCUMENT\_FRAGMENT\_NODE}
\item \member{Node.NOTATION\_NODE}
\end{itemize}
\note{These are defined by the XML DOM, not all of them are used by \plasTeX.}
\end{memberdesc}

\begin{memberdesc}[Node]{parentNode}
refers to the node that contains this node
\end{memberdesc}

\begin{memberdesc}[Node]{previousSibling}
the node in the document that is adjacent to and immediately before this 
node.  If one does not exist, the value is \var{None}.
\end{memberdesc}

\begin{memberdesc}[Node]{nextSibling}
the node in the document that is adjacent to and immediately after this 
node.  If one does not exist, the value is \var{None}.
\end{memberdesc}

\begin{memberdesc}[Node]{ownerDocument}
the node that owner of, and ultimate parent of, all nodes in the document
\end{memberdesc}

\begin{memberdesc}[Node]{textContent}
contains just the text content of this node
\end{memberdesc}

\begin{memberdesc}[Node]{unicode}
specifies a unicode string that could be used in place of the node.
This unicode string will be converted into tokens in the \plasTeX\ 
output stream.
\end{memberdesc}

\begin{memberdesc}[Node]{userdata}
dictionary used for holding user-defined data
\end{memberdesc}


\begin{methoddesc}[Node]{__add__}{other}
create a new node that is the sum of \var{self} and \var{other}.  This
allows you to use nodes in Python statements like: node + other.
\end{methoddesc}

\begin{methoddesc}[Node]{append}{newChild}
adds a new child to the end of the child nodes
\end{methoddesc}

\begin{methoddesc}[Node]{appendChild}{newChild}
same as \method{append}
\end{methoddesc}

\begin{methoddesc}[Node]{cloneNode}{deep=False}
create a clone of the current node.  If \var{deep} is true, then the 
attributes and child nodes are cloned as well.  Otherwise, all references
to attributes and child nodes will be shared between the nodes.
\end{methoddesc}

\begin{methoddesc}[Node]{__cmp__}{other}
same as \method{isEqualNode}, but allows you to compare nodes using the
Python statement: node == other.
\end{methoddesc}

\begin{methoddesc}[Node]{extend}{other}
appends \var{other} to list of children then returns \var{self}
\end{methoddesc}

\begin{methoddesc}[Node]{__getitem__}{i}
returns the child node at the index given by \var{i}.  This allows you to
use Python's slicing syntax to retrieve child nodes: node[i].
\end{methoddesc}

\begin{methoddesc}[Node]{getUserData}{key}
retrieves the data in the \member{userdata} dictionary under the name \var{key}
\end{methoddesc}

\begin{methoddesc}[Node]{hasAttributes}{}
returns a boolean indicating whether or not this node has attributes defined
\end{methoddesc}

\begin{methoddesc}[Node]{hasChildNodes}{}
returns a boolean indicating whether or not the node has child nodes
\end{methoddesc}

\begin{methoddesc}[Node]{__iadd__}{other}
same as \method{extend}.  This allows you to use nodes in Python statements
like: node += other.
\end{methoddesc}

\begin{methoddesc}[Node]{insert}{i, newChild}
inserts node \var{newChild} into position \var{i} in the child nodes list
\end{methoddesc}

\begin{methoddesc}[Node]{insertBefore}{newChild, refChild}
inserts \var{newChild} before \var{refChild} in this node.  If \var{refChild}
is not found, a \exception{NotFoundErr} exception is raised.
\end{methoddesc}

\begin{methoddesc}[Node]{isEqualNode}{other}
indicates whether the given node is equivalent to this one
\end{methoddesc}

\begin{methoddesc}[Node]{isSameNode}{other}
indicates whether the given node is the same node as this one
\end{methoddesc}

\begin{methoddesc}[Node]{__iter__}{}
returns an iterator that iterates over the child nodes.  This allows you to
use Python's \function{iter()} function on nodes.
\end{methoddesc}

\begin{methoddesc}[Node]{__len__}{}
returns the number of child nodes.  This allows you to use Python's 
\function{len()} function on nodes.
\end{methoddesc}

\begin{methoddesc}[Node]{normalize}{}
combine consecutive text nodes and remove comments in this node
\end{methoddesc}

\begin{methoddesc}[Node]{pop}{index=-1}
removes child node and the index given by \var{index}.  If no index is
specified, the last child is removed.
\end{methoddesc}

\begin{methoddesc}[Node]{__radd__}{other}
create a new node that is the sum of \var{other} and \var{self}.  This
allows you to use nodes in Python statements like: other + node.
\end{methoddesc}

\begin{methoddesc}[Node]{replaceChild}{newChild, oldChild}
replaces \var{oldChild} with \var{newChild} in this node.  If \var{oldChild}
is not found, a \exception{NotFoundErr} exception is raised.
\end{methoddesc}

\begin{methoddesc}[Node]{removeChild}{oldChild}
removes \var{oldChild} from this node.  If \var{oldChild} is not found,
a \exception{NotFoundErr} exception is raised.
\end{methoddesc}

\begin{methoddesc}[Node]{__setitem__}{i, node}
sets the item at index \var{i} to \var{node}.  This allows you to use 
Python's slicing syntax to insert child nodes; see the example below.
\begin{verbatim}
mynode[5] = othernode
mynode[6:10] = [node1, node2]
\end{verbatim}
\end{methoddesc}

\begin{methoddesc}[Node]{setUserData}{key, data}
put data specified in \var{data} into the \member{userdata} dictionary under the
name given by \var{key}
\end{methoddesc}

\begin{methoddesc}[Node]{toXML}{}
return an XML representation of the node
\end{methoddesc}


\subsection{DocumentFragment Objects}

\begin{classdesc}{DocumentFragment}{}
A collection of nodes that make up only part of a document.  This is 
mainly used to hold the content of a \LaTeX\ macro argument.
\end{classdesc}


\subsection{Element Objects}

\begin{classdesc}{Element}{}
The base class for all element-type nodes in a document.  Elements generally
refer to nodes created by \LaTeX\ commands and environments.
\end{classdesc}

\begin{methoddesc}[Element]{getAttribute}{name}
returns the attribute specified by \var{name}
\end{methoddesc}

\begin{methoddesc}[Element]{getElementById}{elementId}
retrieve the element with the given ID
\end{methoddesc}

\begin{methoddesc}[Element]{getElementsByTagName}{tagName}
retrieve all nodes with the given name in the node
\end{methoddesc}

\begin{methoddesc}[Element]{hasAttribute}{name}
returns a boolean indicating whether or not the specified attribute exists
\end{methoddesc}

\begin{methoddesc}[Element]{removeAttribute}{name}
removes the attribute \var{name} from the \member{attributes} dictionary
\end{methoddesc}

\begin{methoddesc}[Element]{setAttribute}{name, value}
sets the attribute \var{value} in the \member{attributes} dictionary
using the key \var{name}
\end{methoddesc}


\subsection{Text Objects}

\begin{classdesc}{Text}{}
This is the node type used for all text data in a document object.
Unlike XML DOM text nodes, text nodes in \plasTeX\ are 
not mutable.  This is because they are a subclass of \var{unicode}.
This means that they will respond to all of the standard Python
string methods in addition to the \class{Node} methods and the methods
described below.
\end{classdesc}

\begin{memberdesc}[Text]{data}
the text content of the node
\end{memberdesc}

\begin{memberdesc}[Text]{length}
the length of the text content
\end{memberdesc}

\begin{memberdesc}[Text]{nodeValue}
the text content of the node
\end{memberdesc}

\begin{memberdesc}[Text]{wholeText}
returns the text content from the current text node as well as its siblings
\end{memberdesc}


\subsection{Document Objects}

\begin{classdesc}{Document}{}
The top-level node of a document that contains all other nodes.
\end{classdesc}

\begin{methoddesc}[Document]{createDocumentFragment}{}
instantiate a new document fragment
\end{methoddesc}

\begin{methoddesc}[Document]{createElement}{tagName}
instantiate a new element with the given name
\end{methoddesc}

\begin{methoddesc}[Document]{createTextNode}{data}
instantiate a new text node initialized with \var{data}
\end{methoddesc}

\begin{methoddesc}[Document]{importNode}{importedNode, deep=False}
import a node from another document.  If \var{deep} is true, all nodes
within \var{importedNode} are cloned.
\end{methoddesc}

\begin{methoddesc}[Document]{normalizeDocument}{}
concatenate all consecutive text nodes and remove comments
\end{methoddesc}



\subsection{Command Objects}

\begin{classdesc}{Command}{}
The \class{Command} class is a subclass of \class{Macro}.  This is the class
that should be subclassed when creating Python based macros that correspond
to \LaTeX\ commands.

For more information on the \class{Command} class' API, see the 
\class{Macro} class.
\end{classdesc}


\subsection{Environment Objects}

\begin{classdesc}{Environment}{}
The \class{Environment} class is a subclass of \class{Macro}.  This is the
class that should be subclassed when creating Python based macros that 
correspond to \LaTeX\ environments.  The main difference between the 
processing of \class{Command}s and \class{Environment}s is that the
\method{invoke()} method does special handling of the \LaTeX\ document
context, and the \method{digest()} method absorbs the output stream tokens
that are encapsulated by the \macro{begin} and \macro{end} tokens. 

For more information on the \class{Environment} class' API, see the 
\class{Macro} class.
\end{classdesc}


\subsection{TeXFragment Objects}

\begin{classdesc}{TeXFragment}{}
A fragment of a document.  This class is used mainly to store the contents
of \LaTeX\ macro arguments.
\end{classdesc}

\begin{memberdesc}[TeXFragment]{source}
the \LaTeX\ source representation of the document fragment
\end{memberdesc}


\subsection{TeXDocument Objects}

\begin{classdesc}{TeXDocument}{}
A complete \LaTeX\ document.
\end{classdesc}

\begin{memberdesc}[TeXDocument]{charsubs}
a list of two element tuples containing character substitutions for all
text nodes in a document.  This is used to convert charcter strings
like ``-{}-{}-'' into ``---''.  The first element in each tuple in the
string to replace, the second element is the unicode character or 
sequence to replace the original string with.
\end{memberdesc}

\begin{memberdesc}[TeXDocument]{preamble}
returns the \LaTeX\ source representation of the document preamble 
(i.e. everything before the \macro{begin\{document\}})
\end{memberdesc}

\begin{memberdesc}[TeXDocument]{source}
the \LaTeX\ source representation of the document 
\end{memberdesc}
